\documentclass[a4paper]{article}

\usepackage[T1]{fontenc}	
\usepackage{amsmath}
\usepackage{amssymb}
\usepackage{graphicx}
\usepackage{fancyhdr}
\usepackage{array}

\pagestyle{fancy}
\lhead{Computer Excercise 1}
\rhead{Kristoffer Nordström, Noah Hansson}
\cfoot{\thepage}

\title{Computer Exercise 1 - NEKN83}
\author{Kristoffer Nordström, Noah Hansson}
\date{\today}

\setlength{\parskip}{0.7em}
\setlength{\parindent}{0pt}

\begin{document}

\maketitle

\section{Introduction}
The purpose of this exercise is to implement and evaluate different methods to estimate Value-at-Risk. The data used in the exercise consist of price observations for a given US-stock market portfolio observed between 2006 - 2008. Value at risk is estimated and evaluated for 2008. In the results we conclude that the historical models gives a steadier results, but have harder capturing the changes in volatility compared to the models based on GARCH.

\section{Empirical Results}
The exercise was solved by using Excel and the plots was made with the help of python.

\begin{figure}[htbp]
    \includegraphics[width=\textwidth]{VaR1.png}
    \caption{Observed loss compared to $VaR_{99\%}$ estimates based on historical volatility}
    \label{var1}
\end{figure}

\begin{figure}[htbp]
    \includegraphics[width=\textwidth]{VaR2.png}
    \caption{Observed loss compared to $VaR_{99\%}$ estimates based on EWMA volatility predictions}
    \label{var2}
\end{figure}

\begin{table}[htbp]
	\centering
    \caption{95\% Confidence Interval}
    \vspace{0.2cm}
    \begin{tabular}{|c|c|}
        \hline
        Lower Bound & 0  \\ \hline
        Upper Bound & 6  \\\hline
	\end{tabular}
\end{table}

\begin{table}[htbp]
	\centering
    \caption{Number Of Violations}
    \vspace{0.2cm}
    \begin{tabular}{|m{1.8cm}|*6{m{1.1cm}|}}
        \hline
		& Regular & Regular EWMA & Normal & Normal EWMA & T-Dist & T-Dist EWMA \\ \hline
		Number Of Violations & 22 & 7 & 30 & 7 & 26 & 6 \\\hline
	\end{tabular}
\end{table}


\section{Conclusion}
As seen when comparing figure \ref{var1} and figure \ref{var2}, using an EWMA volatility predictor gives us a better VaR-estimate. In fact, the only VaR estimate that manages to pass the Kupiec test is the EWMA T-distribution estimate. However, both methods have their pros and cons. Even though the EWMA methods better manage to follow the market when volatility spikes, it also has a higher variance in estimates, and tends to over-react some times. This gives a higher VaR-estimate than necessary. When volatility is rather constant the historical volatility estimates give a lower VaR without causing too many violations. \newline
We can also see that the choice of volatility model matters a lot more than the underlying assumption of distribution of the loss function. As such, the first step towards making a more reliable estimator would be to find a better predictor of volatility.

\end{document}
