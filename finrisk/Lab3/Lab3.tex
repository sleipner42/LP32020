\documentclass[a4paper]{article}
\usepackage[bottom=10em]{geometry}

\usepackage[T1]{fontenc}	
\usepackage{amsmath}
\usepackage{amssymb}
\usepackage{graphicx}
\usepackage{fancyhdr}
\usepackage{array}
\usepackage{float}
\usepackage{booktabs}


\pagestyle{fancy}
\lhead{Computer Excercise 2}
\rhead{Kristoffer Nordström \& Noah Hansson}

\title{Computer Exercise 3 - NEKN83}
\author{Kristoffer Nordström \\ kr8245no-s@student.lu.se \and  Noah Hansson \\ no3822ha-s@student.lu.se}
\date{\today}

\setlength{\parskip}{0.7em}
\setlength{\parindent}{0pt}
\setlength{\floatsep}{6pt plus 1.0pt minus 2.0pt}
\setlength{\textfloatsep}{10pt plus 1.0pt minus 2.0pt}

\begin{document}

\maketitle

\section{Introduction}
In this exercise we will implement the Merton model to estimate the default probabilities of a single firm, i.e the probability that the firm will default in the next coming 12 months. For this exercise we will look at the firm Ericsson for five different months during the period 1999-2001, a period with high volatility in the tech industry.

\section{Method}
To estimate the default probability using the Merton Model we need to gather historical data about the firm and market conditions. The data we need is:
\begin{itemize}
    \item Stock prices and the number of outstanding shares
    \item Face value of debt
    \item The risk-free rate
\end{itemize}

In simple terms, the Merton model assumes that the face value of debt is treated as a zero-coupon bond with a duration of one year. The shareholder payoff can then be seen as a call option on the firm's asset value at time $T$, with a strike price of the face value of the debt. The price for this call option would then be equal to the firm's equity. Therefore we can use the Black-Scholes model "in reverse" to find the market value of the firms assets. From there, we can find the probability of default as the probability that the debt is higher than the asset value. We can also express the probability as the distance to default, i.e the amount of standard deviations on the asset values needed to reach default.

Finally, we use these results to look at the implied credit spread of the firm compared to the yearly risk free rate for the period.

\section{Results}

The results for the calculations can be 
\begin{table}[H]
    \centering
    \caption{The calculated variables for each of the five months}
    \label{tab:result}
    \begin{tabular}{lrrrrr}
\toprule
    Unnamed: 9 &    Mar 98 &    Oct 98 &    Sep 99 &    Nov 00 &    Oct 01 \\
\midrule
             E &   317.488 &   309.211 &   487.076 &  1037.131 &   292.778 \\
         $r_f$ &     0.043 &     0.042 &     0.030 &     0.040 &     0.040 \\
             K &   480.876 &   516.875 &   601.411 &   762.700 &   828.700 \\
    $\sigma_A$ &     0.108 &     0.329 &     0.167 &     0.454 &     0.457 \\
             A &   777.968 &   798.113 &  1070.681 &  1764.955 &  1011.898 \\
            d1 &     4.932 &     1.611 &     3.717 &     2.163 &     0.753 \\
            d2 &     4.824 &     1.282 &     3.550 &     1.709 &     0.296 \\
         N(d1) &     1.000 &     0.946 &     1.000 &     0.985 &     0.774 \\
         N(d2) &     1.000 &     0.900 &     1.000 &     0.956 &     0.616 \\
        Solver & 2.039e-18 & 1.143e-20 & 7.680e-14 & 2.430e-09 & 8.957e-09 \\
            DD &     4.824 &     1.282 &     3.550 &     1.709 &     0.296 \\
            PD & 7.030e-07 &     0.100 & 1.929e-04 &     0.044 &     0.384 \\
 Credit Spread & 1.426e-08 &     0.014 & 7.704e-06 & 6.974e-03 &     0.102 \\
\bottomrule
\end{tabular}

\end{table}

\section{Conclusion}

\end{document}