\documentclass[a4paper]{article}

\usepackage[T1]{fontenc}	
\usepackage{amsmath}
\usepackage{amssymb}

\begin{document}
 
\section*{Random Number Generation}

\subsection*{a)}
We are intrested in finding the conditional distribution function of a stochastic variable $X$ given that $X$ is within a given interval, i.e $F_{X|X\in I}(x)$, where I is the closed interval $I = [a,b]$. Using the law of total probability we can express this as
\begin{equation}
    F_{X|\in{I}}(x) = \frac{\int_a^xf_X(x)dx}{c} = \frac{\int_a^xf_X(x)dx}{\int_a^bf_X(x)dx} = \frac{F_X(x)-F_X(a)}{F_X(b)-F_X(a)}
\end{equation}
From this we can easily compute the conditional probability density function
\begin{equation}
    f_{X|\in{I}}(x) = \frac{dF_{X|\in{I}}(x)}{dx} = \frac{f_X(x)}{F_X(b)-F_X(a)}
\end{equation}

\subsection*{b)}
The inverse to the conditional distribution function can be found by expressing it as a composition of functions and finding the inverse of the composition.
\begin{equation}
    F_{X|\in{I}}(x) = (f \circ g)(x) = \frac{F_X(x)-F_X(a)}{F_X(b)-F_X(a)}
\end{equation}
where
\begin{equation}
    g(x) = F_X(x) - F_X(a)
\end{equation}
and
\begin{equation}
    f(x) = \frac{x}{F_X(b) - F_X(a)}
\end{equation}
which gives the inverse
\begin{equation}
    F_{X|\in{I}}^{-1}(x) = (g^{-1} \circ f^{-1})(x) = F_X^{-1}((F_X(b)- F_X(a))x + F_X(a))
\end{equation}
Since the inverse of a probability distribution function is only defined for $x \in [0,1]$ we can then generate samples from the conditional distribution $ F_{X|\in{I}}(x)$ by sampling uniform samples $U[0,1]$ and run them through the inverse distribution $F_{X|\in{I}}^{-1}(x)$.

\subsection*{b2)}
Since we can sample from $F_X$ by generating uniform samples $U[0,1]$ and running them through the filter $F_X^{-1}$, we can express the condition $X \in I$ as the condition that $U \in [F_X(a), F_X(b)]$. This means that we can sample from $F_{X|\in{I}}(x)$ by scaling and shifting $U$ as $(F_X(b) - F_X(a)U + F_X(a))$. Filtering through $F_X^{-1}$ then gives samples of $F_{X|\in I}(x)$.
Therefore, $F_{X| \in I}^{-1}$ can be expressed as 
\begin{equation}
    F_{X|X\in I}^{-1} = F_X^{-1}((F_X(b) - F_X(a))U + F_X(a))
\end{equation}

\newpage
\section*{Power Production of a Wind Turbine}

\subsection*{a) Regular Monte-Carlo sampling}
To calculate an initial estimate of the power production of the Wind Turbine we start by sampling the power production using regular Monte-Carlo sampling. We begin by generating $10^5$ samples of wind distributed by a Weibull distribution. We then calculate the generated power for each wind sample and then find the average and the variance for each month. The estimate is calculated as follows:
\begin{equation}
    \tau = \mathbb{E}[\phi(X)] = \frac{1}{N}\sum_{i = 1}^N\phi(X_i)
\end{equation}
And since the sample size is large we can also assume that the central limit theorem applies, giving us:
\begin{equation}
    \mathbb{V}[\sqrt{N}(\tau_N-\tau)] = N\mathbb{V}[\tau_N-\tau] = \mathbb{V}[\phi(X)]
\end{equation}
In this case the stochastic variable $X$ is the wind speed and $\phi(X)$ is the power curve function for the wind turbine.

For each month we then get the following 95\% confidence intervals of the estimates.

\begin{center}
    \begin{table}
        \caption{Monte Carlo estimates and confidence intervals of power production for each month of the year}
        \label{MCresults}
        \begin{tabular}{|c|| c c c c c c c c c c c c ||}
            \hline
            Month & Jan & Feb & Mar & Apr & May & Jun & Jul & Aug & Sep & Oct & Nov & Dec \\
            \hline\hline
            $\tau$ & 0 & 0 & 0 & 0 & 0 & 0 & 0 & 0 & 0 & 0 & 0 & 0 \\
            \hline
            $\mathbb{V}[\phi(X)]$ & 0 & 0 & 0 & 0 & 0 & 0 & 0 & 0 & 0 & 0 & 0 & 0\\
            \hline
            $I_{95\%}$ & 0 & 0 & 0 & 0 & 0 & 0 & 0 & 0 & 0 & 0 & 0 & 0\\
            \hline

        \end{tabular}
    \end{table}
\end{center}

\subsection*{b) Importance Sampling}
To reduce the variance of the Monte-Carlo estimate we will here apply the importance sampling method. When we generate random wind samples we sometimes get samples of wind speed that give us a power production of zero. These samples give do not help finding a better estimate of the average power, as they only increase the variance of the estimator. Instead, we try to find an instrumental density g where we then draw our wind samples from. This reduces the amount of "bad" samples and reduces the variance of the estimator. \noindent To find an estimate of the average power production we use the following calculations, where the density $g(x)$ is a density such as $g(x) = 0 \rightarrow \phi(x)f(x) = 0$
\begin{equation}
    \tau = \mathbb{E}_f(\phi(X)) = 
\end{equation}

\subsection*{c) Antithetic sampling}


\section*{Power production of two wind turbines}

\subsection*{a)}
In order to find the expectation of the sum of power production, $\mathbb{E}[P(v_1) + P(v_2)]$ we can simply find the sum of expectations, i.e $\mathbb{E}[P(v_1)] + \mathbb{E}[P(v_2)]$. Since both stochastic variables have the same marginal distribution we can simply express the problem as $\mathbb{E}[P(v_1) + P(v_2)] = 2\mathbb{E}[P(v)]$ and find a confidence interval as in section 2.

Using the yearly average parameters of $\lambda = X$ and $k = Y$ we then crude Monte Carlo sampling to find a 95\% confidence interval of:

\subsection*{b)}
To find the covariance of the power production we can use the calculation
\begin{equation}
     \mathbb{C}[P(V_1), P(V_2)] = \mathbb{E}[P(V_1)P(V_2)] - \mathbb{E}[P(V_1)]\mathbb{E}[P(V_2)]
\end{equation}
Since we already know the expected power production of one wind turbine we now only need to find the expectation of the product of the power production. Since the power production is identically distributed for both turbines we can instead find the expectation of the squared power production. Here we will use importance sampling to find an accurate estimate.

\subsection*{c)}
The variance of the combined power production can be computed using the following calculations
\begin{equation}
    \mathbb{V}[P(V_1) + P(V_2)] = \mathbb{V}[P(V_1)] + \mathbb{V}[P(V_1)] + 2\mathbb{C}[P(V_1), P(V_2)]
\end{equation}
Since we have already found all these values we simply plug them in to get the a variance of $X$, and a standard deviation of $\sqrt{X}$



\end{document}
