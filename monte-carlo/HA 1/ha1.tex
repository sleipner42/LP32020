\documentclass[a4paper]{article}

\usepackage[T1]{fontenc}	
\usepackage{amsmath}
\usepackage{amssymb}

\begin{document}
 
\section*{Random Number Generation}

\subsection*{1)}

\subsubsection*{a)}

\begin{equation}
     \begin{gathered}
        F_{X|x\in{I}}(x) = P(X < x | X \in I) \\
        = P(F_X^{-1}(U) < x | F_X^{-1}(U) \in I) \\
        = P(F_X^{-1}((F_X(b)-F_X(a))U+F_X(a)) < x) \\
        = P((F_X(b)-F_X(a))U + F_X(a) < F_X(x)) \\
        = P(U < \frac{F_X(x)-F_X(a)}{F_X(b)-F_X(a)})
        = \frac{F_X(x)-F_X(a)}{F_X(b)-F_X(a)}
     \end{gathered}
\end{equation}

Annat sätt att räkna

\begin{equation}
    F_{X|\in{I}}(x) = \frac{\int_a^xf_X(x)dx}{c} = \frac{\int_a^xf_X(x)dx}{\int_a^bf_X(x)dx} = \frac{F_X(x)-F_X(a)}{F_X(b)-F_X(a)}
\end{equation}

\begin{equation}
    f_{X|\in{I}}(x) = \frac{dF_{X|\in{I}}(x)}{dx} = \frac{f_X(x)}{F_X(b)-F_X(a)}
\end{equation}

\subsubsection*{b)}

\begin{equation}
    F_{X|\in{I}}^{-1}(x) = F_X^{-1}((F_X(b)- F_X(a))x + F_X(a))
\end{equation}
\newpage
\section*{Power Production of a Wind Turbine}

\subsection*{a) Regular Monte-Carlo sampling}
To calculate an initial estimate of the power production of the Wind Turbine we start by sampling the power production using regular Monte-Carlo sampling. We begin by generating $10^5$ samples of wind distributed by a Weibull distribution. We then calculate the generated power for each wind sample and then find the average and the variance for each month. The estimate is calculated as follows:
\begin{equation}
    \tau = \mathbb{E}[\phi(X)] = \frac{1}{N}\sum_{i = 1}^N\phi(X_i)
\end{equation}
And since the sample size is large we can also assume that the central limit theorem applies, giving us:
\begin{equation}
    \mathbb{V}[\sqrt{N}(\tau_N-\tau)] = N\mathbb{V}[\tau_N-\tau] = \mathbb{V}[\phi(X)]
\end{equation}
In this case the stochastic variable $X$ is the wind speed and $\phi(X)$ is the power curve function for the wind turbine.

For each month we then get the following 95\% confidence intervals of the estimates.

\begin{center}
    \begin{table}
        \caption{Monte Carlo estimates and confidence intervals of power production for each month of the year}
        \label{MCresults}
        \begin{tabular}{|c|| c c c c c c c c c c c c ||}
            \hline
            Month & Jan & Feb & Mar & Apr & May & Jun & Jul & Aug & Sep & Oct & Nov & Dec \\
            \hline\hline
            $\tau$ & 0 & 0 & 0 & 0 & 0 & 0 & 0 & 0 & 0 & 0 & 0 & 0 \\
            \hline
            $\mathbb{V}[\phi(X)]$ & 0 & 0 & 0 & 0 & 0 & 0 & 0 & 0 & 0 & 0 & 0 & 0\\
            \hline
            $I_{95\%}$ & 0 & 0 & 0 & 0 & 0 & 0 & 0 & 0 & 0 & 0 & 0 & 0\\
            \hline

        \end{tabular}
    \end{table}
\end{center}

\subsection*{b) Importance Sampling}
To reduce the variance of the Monte-Carlo estimate we will here apply the importance sampling method. When we generate random wind samples we sometimes get samples of wind speed that give us a power production of zero. These samples give do not help finding a better estimate of the average power, as they only increase the variance of the estimator. Instead, we try to find an instrumental density g where we then draw our wind samples from. This reduces the amount of "bad" samples and reduces the variance of the estimator. \\
\newline
\noindent To find an estimate of the average power production we use the following calculations, where the density $g(x)$ is a density such as $g(x) = 0 \rightarrow \phi(x)f(x) = 0$
\begin{equation}
    \tau = \mathbb{E}_f(\phi(X)) = 
\end{equation}

\subsection*{c) Antithetic sampling}




\end{document}
