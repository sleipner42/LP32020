\documentclass[a4paper]{article}

\usepackage[T1]{fontenc}	
\usepackage{amsmath}
\usepackage{amssymb}
\usepackage{fancyhdr}
\usepackage{booktabs}
\usepackage{graphicx}
\usepackage{float}
\usepackage[margin=1in]{geometry}

\pagestyle{fancy}
\rhead{Home Assignment 2}
\lhead{Noah Hansson \& Kristoffer Nordström}

\title{Home Assignment 2 - FMSN50}
\author{Kristoffer Nordström, Noah Hansson}\author{Kristoffer Nordström \\ kr8245no-s@student.lu.se \and  Noah Hansson \\ no3822ha-s@student.lu.se}
\date{\today}


\setlength{\parskip}{0.7em}
\setlength{\parindent}{0pt}
\setlength{\floatsep}{6pt plus 1.0pt minus 2.0pt}
\setlength{\textfloatsep}{10pt plus 1.0pt minus 2.0pt}

\begin{document}
\maketitle
\newpage

\section*{1)}

All possible Self Avoiding Walks (SAW), $S_{n+m}(d)$, could be constructed by combining n-step SAWs with m-step SAWs, but all of these combination will not be self avoiding and thus is

\begin{equation}
    c_{n+m}(d)\leq c_n(d)c_m(d)
\end{equation}

\section*{2)}
From the result in section 1 we can prove that the logarithm of $c_n$ is subadditive:
\begin{equation}
    c_{n+m}(d) \leq c_n(d)c_m(d) \rightarrow log(c_{n+m}) \leq log(c_n(d)c_m(d)) = log(c_n(d)) + log(c_m(d))
\end{equation}

 Therefore, we can use \textit{Fekete's Lemma} to prove that $\lim_{n\rightarrow\infty} \frac{log(c_n(d))}{n}$ exists. Since $e(x)$ is a well defined function for all $x$, we know that the limit $\lim_{n\rightarrow\infty} exp(\frac{log(c_n(d))}{n}) = \lim_{n\rightarrow\infty} c_n(d)^{1/n}$ exists. 


\end{document}
