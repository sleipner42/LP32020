\documentclass[a4paper]{article}

\usepackage[T1]{fontenc}	
\usepackage{amsmath}
\usepackage{amssymb}
\usepackage{fancyhdr}
\usepackage{booktabs}
\usepackage{graphicx}
\usepackage{float}
\usepackage[margin=1in]{geometry}

\pagestyle{fancy}
\rhead{Home Assignment 3}
\lhead{Noah Hansson \& Kristoffer Nordström}

\title{Home Assignment 3 - FMSN50}
\author{Kristoffer Nordström, Noah Hansson}\author{Kristoffer Nordström \\ kr8245no-s@student.lu.se \and  Noah Hansson \\ no3822ha-s@student.lu.se}
\date{\today}


\setlength{\parskip}{0.7em}
\setlength{\parindent}{0pt}
\setlength{\floatsep}{6pt plus 1.0pt minus 2.0pt}
\setlength{\textfloatsep}{10pt plus 1.0pt minus 2.0pt}

\begin{document}
\maketitle
\newpage

\section*{Coal mine disasters - Constructing a complex MCMC algorithm}


\subsection*{1)}
We are interested in sampling from the posterior distribution $f(\lambda, \theta, t | \tau)$ using a hybrid MCMC algorithm. To do this, we must first find the marginal posterior distributions:
\begin{itemize}
    \item $f(\theta | \tau, \lambda, t)$
    \item $f(t | \tau, \lambda, \theta)$
    \item $f(\lambda | \tau, \theta, t)$
\end{itemize}

To do this we begin by reducing the dimension of the posterior using the Bayes' theorem and the chain rule, giving us:
\begin{equation}
    f(\lambda, \theta, t | \tau) \propto f(\tau|\lambda, \theta, t)f(\lambda, \theta,t) = f(\tau|\lambda,t)f(t)f(\theta)f(\lambda|\theta)
\end{equation}
From this, we can find the marginal posterior distributions up to a normalizing constant:
\begin{equation}
    \begin{cases}
        f(\theta | \tau, \lambda, t) \propto f(\theta)f(\lambda|\theta) \\
        f(t | \tau, \lambda, \theta) \propto f(t)f(\tau|\lambda,t) \\
        f(\lambda | \tau, \theta, t) \propto f(\lambda|\theta)f(\tau|\lambda,\theta)
    \end{cases}
\end{equation}

From these marginal distributions we can attempt to find a general distribution to sample the variables. For $\lambda$ this gives us:
\begin{equation}
    \begin{gathered}
        f(\lambda | \tau, \theta, t) \propto f(\lambda|\theta)f(\tau|\lambda,\theta) = \\ = \theta^2\lambda\exp(-\theta\lambda-\sum_{i=1}^d(t_{i+1}-t_i))\prod_{t=1}^d\lambda_i^{n_i(\tau)}
    \end{gathered}
\end{equation}
Since lambda is a vector of intensities $\lambda_i$ we can sample each individual lambda as 
\begin{equation}
    f(\lambda_i|\theta)f(\tau|\lambda_i,t) = \theta^2\lambda_i^{n_i(\tau)+1}\exp(-\lambda_i(\theta + (t_{i+1}-t_i)))
\end{equation}
From this expression we can conclude that the marginal posterior for $\lambda$ is gamma distributed such that:
\begin{equation}
    \lambda_i \sim \Gamma(2+n_i(\tau), \theta + (t_{i+1} - t_i))
\end{equation}

Similarly, for $\theta$:
\begin{equation}
    \begin{gathered}
        f(\theta|, \tau, \lambda, t) \propto f(\theta)f(\lambda|\theta) = f(\theta)\prod_{i=1}^d f(\lambda_i|\theta)= \\
        = \Psi^2\theta\exp(-\theta\Psi)\prod_{i=1}^d(\theta^2\lambda_i\exp(-\lambda_i\theta)) = \\
        = \Psi^2\theta^{2d+1}\exp(-\theta(\Psi+\sum_{i=1}^d\lambda_i))\prod_{i=1}^d\lambda_i
    \end{gathered}
\end{equation}
Here we can conclude that the marginal posterior for $\theta$ is gamma distributed such that:
\begin{equation}
    \theta \sim \Gamma(2+2d, \Psi+\sum_{i=1}^d\lambda_i)
\end{equation}

Unfortunately the marginal distribution of $t$ cannot be sampled from a known distribution. This is why we need a hybrid MCMC sampler to sample from the posterior. The hybrid sampler can use the Gibbs algorithm for $\theta$ and $\lambda$ since they can be sampled from, but we will need to use the Metropolis Hastings algorithm for $t$.

\subsection*{2)}

To sample from $f(\lambda, \theta, t | \tau)$ we need to construct a hybrid MCMC sampler using both the Gibbs sampler and the Metropolis-Hastings algorithm. We begin by sampling a random number $\theta$ from its prior distribution, and then sample the random vector \textbf{$\lambda$} from its prior distribution depending on $\theta$. We also create breakpoints $t$ spaced evenly in the interval [1658, 1980]. Then the algorithm loops through following steps:
\begin{enumerate}
    \item Sample $\theta_{k+1}$ from $f(\theta|\tau, \lambda_k, t_k)$
    \item Sample $\lambda_{k+1}$ from $f(\lambda|\tau, \theta_{k+1}, t_k)$
    \item Propose $t_{k+1}$ from a random walk proposal
    \item Accept or reject $t_{k+1}$
\end{enumerate}

Here we use a random walk proposal for the breakpoints $t$.

\newpage

\section{Parametric bootstrap for the 100-year Atlantic wave}


\end{document}